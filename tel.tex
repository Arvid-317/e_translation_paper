\section{Temporal Equilibrium Logic $TEL_f$}\label{sec:telf}

Temporal Equilibrium Logic can be defined from Temporal Here and There, as described in \cite{cakascsc18a}.

\begin{itemize}
  \item \emph{Temporal formulas} are defined as follow :
  $$\psi ::= a | \bot | \psi_1\otimes\psi_2 | \previous\psi | \psi_1\since\psi_2 | \psi_1\trigger\psi_2 | \next\psi | \psi_1\until\psi_2 | \psi_1\release\psi_2 $$
  where $a$ is an atom and $\otimes$ is any binary Boolean connective $\otimes \in \{\rightarrow, \wedge, \vee\}$.
  \item
\end{itemize}

\section{Translation}\label{sec:trans}

\begin{itemize}
  \item In Action Langage $\mathcal{E}$, only fluents are present in the trace. Actions only serves to define initiations and terminations points, which are properties of states of the trace.

  To translate that, we also add the actions and initiation/termination points of each fluent as atoms.
  \item The propositions can be translated into temporal formulas as follow :
  \begin{itemize}
    \item \ref{c1} can be translated into $\alwaysF((a\land c)\rightarrow initiation(f))$.
    \item \ref{c2} can be translated into $\alwaysF((a\land c)\rightarrow termination(f))$.
    \item \ref{h} can be translated into $\next^{t}(\neg l\rightarrow \bot)$.
    \item \ref{t} can be translated into $\next^{t}a$.
  \end{itemize}
\end{itemize}
