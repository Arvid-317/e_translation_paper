\section{Translation}\label{sec:translation}

This section will describe a translation of the Action Langage E (which closely resemble Event Calculus in multiples ways) into Temporal Logic.
To be more precise, into $LTL_f$ and $TEL_f$.

The goal is to start from a domain language $E=<\Pi,\preceq,\Delta,\Phi>$ and a domain description $D=<\gamma,\eta,\tau>$, and to get a set of temporal propositions $A$ and formulas $F$, such that all formulas of $F$ are true for a finite traces $\pi$ if and only if the corresponding interpretation $H$ of $E$ is a model of $D$.

\subsection{$LTL_f$}

\subsubsection{Translating the Domain Language}

To make the translation, we need to start from a domain language for which $\Pi$ is a discrete set of timepoint, and $\preceq$ is a total ordering of $\Pi$.

The set $A$ of the following propositional atoms, and only them.

\begin{itemize}
  \item $\Pi$ and $\preceq$ can be represented by a trace $\pi$ of the same size. (with a $\next^{t_f}Tail$ or something like that)
  \item For all $A$ in $\Delta$ there is a propositional variable $a$.
  \item For all $F$ in $\Phi$ there is a propositional variable $f$ and two propositional variables $initiation(f)$ and $termination(f)$.
\end{itemize}

\subsubsection{Axioms}

\begin{itemize}
  \item Common Inertia
  %$$\alwaysF(f\rightarrow termination(f) \release f)$$
  %$$\alwaysF(\neg f\rightarrow initiation(f) \release \neg f)$$
  $$\alwaysF(f \bigwedge \neg termination(f) \rightarrow \wnext f)$$
  $$\alwaysF(\neg f \bigwedge \neg initiation(f) \rightarrow \wnext\neg f)$$
  \item Initiation $$\alwaysF(initiation(f)\rightarrow\wnext f)$$
  \item Termination $$\alwaysF(termination(f)\rightarrow\wnext \neg f)$$
%  \item Initiation Inertia $$\alwaysF(initiation(f_i)\rightarrow\next (termination(f_i) \mathbin{\bm{\mathsf{R}}} f_i))$$
%  \item Termination Inertia $$\alwaysF(termination(f_i)\rightarrow\next (initiation(f_i) \mathbin{\bm{\mathsf{R}}} \neg f_i))$$
\end{itemize}

\subsubsection{Translationg the Domain Definition}

\begin{itemize}
  \item c-proposition\footnote{c-proposition's translation also place initiation and termination point for the temporal logic}
  \begin{itemize}
    \item "$a$ initiates $f$ when $c$" become "$\alwaysF((a\bigwedge c)\rightarrow initiation(f))$"
    \item "$a$ terminates $f$ when $c$" become "$\alwaysF((a\bigwedge c)\rightarrow termination(f))$"
  \end{itemize}
  \item h-proposition
  \begin{itemize}
    \item "$f$ holds-at $t_i$" become "$\next^{i}f$"
    \item "$\neg f$ holds-at $t_i$" become "$\next^{i} \neg f$"
  \end{itemize}
  \item t-proposition
  \begin{itemize}
    \item "$a$ happens-at $t_i$" become "$\next^{i}a$"
  \end{itemize}
\end{itemize}

\subsection{Example}

Let's take a domain language $E=<\Pi,\preceq,\Delta,\Phi>$ with :
\begin{itemize}
  \item $\Pi$ composed of 5 timepoints $t_0$ to $t_4$
  \item $\Delta$ only composed of the action $switch$
  \item $\Phi$ only composed of the action $light$
\end{itemize}

Let's take a domain description $D=<\gamma,\eta,\tau>$ with :
\begin{itemize}
  \item $\gamma$ composed of two c-proposition :
  \begin{itemize}
    \item $switch$ initiates $light$ when $\neg light$.
    \item $switch$ terminates $light$ when $light$.
  \end{itemize}
  \item $\eta$ composed of two h-proposition :
  \begin{itemize}
    \item $switch$ happens-at $t_1$.
    \item $switch$ happens-at $t_3$.
  \end{itemize}
  \item $\tau$ composed of no t-proposition.
\end{itemize}

Once translated we get :
\begin{itemize}
  \item The length of the trace $length(\pi)=5$.
  \item The set of propositions $A$ composed of $switch$, $light$, $initiation(light)$ and $termination(light)$.
  \item The set of formulas $F$ composed of :
  \begin{itemize}
    \item The formulas taken from the propositions :
    \item $\alwaysF(switch\bigwedge \neg light)\rightarrow initiation(light)$
    \item $\alwaysF(switch\bigwedge light)\rightarrow termination(light)$
    \item $\next^{1}switch$
    \item $\next^{3}switch$
    \item The formulas taken from the inertia axioms :
    \item $\alwaysF(light\rightarrow\next (termination(light) \release light))$
    \item $\alwaysF(\neg light\rightarrow\next (initiation(light) \release \neg light))$
    \item $\alwaysF(initiation(light)\rightarrow\next light)$
    \item $\alwaysF(termination(light)\rightarrow\next \neg light)$
  \end{itemize}
\end{itemize}

We can see that the original domain description had two models:
\begin{itemize}
  \item $H(light,t_0)=H(light,t_1)=H(light,t_4)=true$ and $H(light,t_2)=H(light,t_3)=false$
  \item $H(light,t_0)=H(light,t_1)=H(light,t_4)=false$ and $H(light,t_2)=H(light,t_3)=true$
\end{itemize}

Likewise, the resulting set of formulas only has two traces that respect them:
\begin{itemize}
  \item $\{light\}\{light,switch,termination(light)\}\{\}\{switch,initiation(light)\}\{light\}$
  \item $\{\}\{switch,initiation(light)\}\{light\}\{light,switch,termination(light)\}\{\}$
\end{itemize}

\subsection{Improvement}




\section{Todo}

\begin{itemize}
  \item Reduce the trace size? Do we need all timepoints, or only the ones with actions?
  \item $LTL_f$ into $TEL$ ... into $telingo$
  \item actions with duration (which probably will need metric temporal logic)
\end{itemize}
