\section{Translation}\label{sec:translation}

This section will describe a translation of the Action Langage E (which closely resemble Event Calculus in multiples ways) into Temporal Logic.
To be more precise, into $LTL_f$ and $TEL_f$.

The goal is to start from a domain language $E=<\Pi,\preceq,\Delta,\Phi>$ and a domain description $D=<\gamma,\eta,\tau>$, and to get a set of temporal propositions $A$ and formulas $F$, such that all formulas of $F$ are true for a finite traces $\pi$ if and only if the corresponding interpretation $H$ of $E$ is a model of $D$.

\subsection{$LTL_f$}

\subsubsection{Translating the Domain Language}

To make the translation, we need to start from a domain language for which $\Pi$ is a discrete set of timepoint, and $\preceq$ is a total ordering of $\Pi$.

The set $A$ is composed of the following propositional atoms, and only them.

\begin{itemize}
  %\item $\Pi$ and $\preceq$ can be represented by a trace $\pi$ of the same size. (with a $\next^{t_f}Tail$ or something like that)
  \item For all $Ac$ in $\Delta$ there is a propositional variable $a$.
  \item For all $Fl$ in $\Phi$ there is the propositional variables $f$, $initiation(f)$ and $termination(f)$.
\end{itemize}

\subsubsection{Axioms}

\begin{description}
  \item[Common Inertia]
  For every $Fl$ in $\Phi$ there is theses two formulas in $F$:
  \begin{equation}\tag{$A.1$}
    \alwaysF(f\rightarrow termination(f) \release f)
  \end{equation}
  \begin{equation}\tag{$A.2$}
    \alwaysF(\neg f\rightarrow initiation(f) \release \neg f)
  \end{equation}
  %$$\alwaysF(f \bigwedge \neg termination(f) \rightarrow \wnext f)$$
  %$$\alwaysF(\neg f \bigwedge \neg initiation(f) \rightarrow \wnext\neg f)$$
  \item[Initiation]
  For every $Fl$ in $\Phi$ there is this formula in $F$:
  \begin{equation}\tag{$A.3$}
    \alwaysF(initiation(f)\rightarrow\wnext f)
  \end{equation}
  \item[Termination]
  For every $Fl$ in $\Phi$ there is this formula in $F$:
  \begin{equation}\tag{$A.4$}
    \alwaysF(termination(f)\rightarrow\wnext \neg f)
  \end{equation}
%  \item Initiation Inertia $$\alwaysF(initiation(f_i)\rightarrow\next (termination(f_i) \mathbin{\bm{\mathsf{R}}} f_i))$$
%  \item Termination Inertia $$\alwaysF(termination(f_i)\rightarrow\next (initiation(f_i) \mathbin{\bm{\mathsf{R}}} \neg f_i))$$
\end{description}

\subsubsection{Translating the Domain Definition}

\begin{description}
  \item[c-proposition]\footnote{c-proposition's translation also place initiation and termination point for the temporal logic}
  \begin{itemize}
    \item For every “$Ac$ initiates $Fl$ when $C$” in $\gamma$ there is theses two formulas in $F$:
    \begin{equation}\tag{$P_c.1$}
      \alwaysF((a\bigwedge c)\rightarrow initiation(f))
    \end{equation}
    \begin{equation}\tag{$P_c.2$}
      \alwaysF(initiation(f)\rightarrow (a\bigwedge c))
    \end{equation}
    \item For every “$Ac$ terminates $Fl$ when $C$” in $\gamma$ there is theses two formulas in $F$:
    \begin{equation}\tag{$P_c.3$}
      \alwaysF((a\bigwedge c)\rightarrow termination(f))
    \end{equation}
    \begin{equation}\tag{$P_c.4$}
      \alwaysF(termination(f)\rightarrow (a\bigwedge c))
    \end{equation}
  \end{itemize}
  \item[h-proposition] $ $
  \begin{itemize}
    \item For every “$Fl$ holds-at $t_i$” in $\eta$ there is this formula in $F$:
    \begin{equation}\tag{$P_h.1$}
      \next^{i}f
    \end{equation}
    \item For every “$\neg Fl$ holds-at $t_i$” in $\eta$ there is this formula in $F$:
    \begin{equation}\tag{$P_h.2$}
      \next^{i} \neg f
    \end{equation}
  \end{itemize}
  \item[t-proposition] $ $
  \begin{itemize}
    \item For every “$Ac$ happens-at $t_i$” in $\tau$ there is this formula in $F$:
    \begin{equation}\tag{$P_t.1$}
      \next^{i}a
    \end{equation}
    \item For all $Ac$ in $\Delta$ and all $t_i$ in $\Pi$, if “$Ac$ happens-at $t_i$” is not in $\tau$ there is this formula in $F$:
    \begin{equation}\tag{$P_t.2$}
      \next^{i}\neg a
    \end{equation}
      This is needed to ensure completion of the actions.

  \end{itemize}
\end{description}

\subsection{Alternatives}

\subsubsection{Combined Axioms}

To be closer to the semantics of the event calculus (the formalism that action language E want to emulate), we can describe Inertia and Initiation/Termination in one go like this:
\begin{description}
  \item[Initiation]
  For every $Fl$ in $\Phi$ there is this formula in $F$:
  \begin{equation}\tag{$CA.1$}
    \alwaysF(initiation(f)\rightarrow\wnext(termination(f) \release f))
  \end{equation}
  \item[Termination]
  For every $Fl$ in $\Phi$ there is this formula in $F$:
  \begin{equation}\tag{$CA.2$}
    \alwaysF(termination(f)\rightarrow\wnext(initiation(f) \release \neg f))
  \end{equation}
\end{description}

\subsubsection{SEC Axioms}

\begin{itemize}
  \item
    For every $Fl$ in $\Phi$ there is this formula in $F$:
    \begin{equation}\tag{$SA.1$}
      \alwaysF(((\previous initiation(f)) \trigger \neg termination(f)) \rightarrow f)
    \end{equation}
  \item
    For every $Fl$ in $\Phi$ there is this formula in $F$:
    \begin{equation}\tag{$SA.2$}
      \alwaysF(((\previous termination(f)) \trigger \neg initiation(f)) \rightarrow \neg f)
    \end{equation}
\end{itemize}

The $(\previous initiation(f)) \trigger \neg termination(f)$ used here represent the negation of the concept of $clipped$ used in EC.
Respectively, $(\previous termination(f)) \trigger \neg initiation(f)$ represent the negation of the concept of $unclipped$.

\subsubsection{Iteratifs Axioms}

In the the Discrete Event Calculus, we only describe a time-point depending of the previous one.

\begin{itemize}
  \item
    For every $Fl$ in $\Phi$ there is this formula in $F$:
    \begin{equation}\tag{$IA.1$}
      \alwaysF((f \bigwedge \neg termination(f)) \rightarrow f)
    \end{equation}
  \item
    For every $Fl$ in $\Phi$ there is this formula in $F$:
    \begin{equation}\tag{$IA.2$}
      \alwaysF((\neg f \bigwedge \neg initiation(f)) \rightarrow \neg f)
    \end{equation}
\end{itemize}

% “…”
