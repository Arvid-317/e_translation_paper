\section{Translation into $LTL_f$}\label{sec:translation}

\subsection{Translating the Domain Language}\label{sec:trans_dom_ltl}

To make the translation, we need to start from a domain language for which $\Pi$ is a discrete set of timepoint, and $\preceq$ is a total ordering of $\Pi$.

The set $S_a$ is composed of the following propositional atoms:

\begin{itemize}
  %\item $\Pi$ and $\preceq$ can be represented by a trace $\pi$ of the same size. (with a $\next^{t_f}Tail$ or something like that)
  \item For all $a$ in $\Delta$ there is a propositional variable $a$.
  \item For all $f$ in $\Phi$ there are the propositional variables $f$, $initiation(f)$ and $termination(f)$.
\end{itemize}

\subsection{Axioms}\label{sec:trans_ax_ltl}

\begin{description}
  \item[Inertia]
  For every $f$ in $\Phi$ we add in $S_f$:
  \begin{equation}\tag{$A.1$}
    \alwaysF(f\rightarrow termination(f) \release f)
  \end{equation}
  \begin{equation}\tag{$A.2$}
    \alwaysF(\neg f\rightarrow initiation(f) \release \neg f)
  \end{equation}
  %$$\alwaysF(f \bigwedge \neg termination(f) \rightarrow \wnext f)$$
  %$$\alwaysF(\neg f \bigwedge \neg initiation(f) \rightarrow \wnext\neg f)$$
  \item[Initiation]
  For every $f$ in $\Phi$ we add in $S_f$:
  \begin{equation}\tag{$A.3$}
    \alwaysF(initiation(f)\rightarrow\wnext f)
  \end{equation}
  \item[Termination]
  For every $f$ in $\Phi$ we add in $S_f$:
  \begin{equation}\tag{$A.4$}
    \alwaysF(termination(f)\rightarrow\wnext \neg f)
  \end{equation}
%  \item Initiation Inertia $$\alwaysF(initiation(f_i)\rightarrow\next (termination(f_i) \mathbin{\bm{\mathsf{R}}} f_i))$$
%  \item Termination Inertia $$\alwaysF(termination(f_i)\rightarrow\next (initiation(f_i) \mathbin{\bm{\mathsf{R}}} \neg f_i))$$
\end{description}

\subsection{Translating the Domain Definition}\label{sec:trans_def_ltl}

\begin{description}
  \item[c-proposition]\footnote{c-proposition's translation also place initiation and termination point for the temporal logic}
  \begin{itemize}
    \item For every “$a$ initiates $f$ when $c$” in $\gamma$ we add in $S_f$:
    \begin{equation}\tag{$P_c.1$}
      \alwaysF((a\land c)\rightarrow initiation(f))
    \end{equation}
    \item For every $f$ in $\Phi$ we add in $S_f$:
    \begin{equation}\tag{$P_c.2$}
      \alwaysF(\bigwedge(\neg a_i\lor \neg c_i)\rightarrow \neg initiation(f))
    \end{equation}
    \item For every “$a$ terminates $f$ when $c$” in $\gamma$ we add in $S_f$:
    \begin{equation}\tag{$P_c.3$}
      \alwaysF((a\land c)\rightarrow termination(f))
    \end{equation}
    \item For every $f$ in $\Phi$ we add in $S_f$:
    \begin{equation}\tag{$P_c.4$}
      \alwaysF(\bigwedge(\neg a_i\lor \neg c_i)\rightarrow \neg termination(f))
    \end{equation}
%    \begin{equation}\tag{$P_c.4$}
%      \alwaysF(termination(f)\rightarrow (a\bigwedge c))
%    \end{equation}
  \end{itemize}
  \item[h-proposition] $ $
  \begin{itemize}
    \item For every “$f$ holds-at $t_i$” in $\eta$ we add in $S_f$:
    \begin{equation}\tag{$P_h.1$}
      \next^{i}f
    \end{equation}
    \item For every “$\neg f$ holds-at $t_i$” in $\eta$ we add in $S_f$:
    \begin{equation}\tag{$P_h.2$}
      \next^{i} \neg f
    \end{equation}
  \end{itemize}
  \item[t-proposition] $ $
  \begin{itemize}
    \item For every “$a$ happens-at $t_i$” in $\tau$ we add in $S_f$:
    \begin{equation}\tag{$P_t.1$}
      \next^{i}a
    \end{equation}\label{sec:trans_def_ltl}
    \item For all $a$ in $\Delta$ and all $t_i$ in $\Pi$, if “$a$ happens-at $t_i$” is not in $\tau$ we add in $S_f$:
    \begin{equation}\tag{$P_t.2$}
      \next^{i}\neg a
    \end{equation}
      This is needed to ensure completion of the actions.
  \end{itemize}
\end{description}

\subsection{Alternative Axiomatizations}

In this section, I wanted to explore alternative ways to express the axioms of $E$.

\subsubsection{SEC Axioms}\label{sec:sec}

\begin{itemize}
  \item
    For every $f$ in $\Phi$ we add in $S_f$:
    \begin{equation}\tag{$SA.1$}
      \alwaysF(((\previous \neg termination(f)) \since f) \rightarrow f)
    \end{equation}
  \item
    For every $f$ in $\Phi$ we add in $S_f$:
    \begin{equation}\tag{$SA.2$}
      \alwaysF(((\previous \neg initiation(f)) \since \neg f) \rightarrow \neg f)
    \end{equation}
\end{itemize}

Those two formulas tries to adapt the concept of $clipped$ and $unclipped$ used in Event Calculus.

\subsubsection{Iteratifs Axioms}

In the the Discrete Event Calculus, we only describe a time-point depending of the previous one.

\begin{itemize}
  \item
    For every $f$ in $\Phi$ we add in $S_f$:
    \begin{equation}\tag{$IA.1$}
      \alwaysF((f \land \neg termination(f)) \rightarrow f)
    \end{equation}
  \item
    For every $f$ in $\Phi$ we add in $S_f$:
    \begin{equation}\tag{$IA.2$}
      \alwaysF((\neg f \land \neg initiation(f)) \rightarrow \neg f)
    \end{equation}
\end{itemize}

% “…”
