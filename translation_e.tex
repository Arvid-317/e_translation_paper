\section{Translation into $TEL$}\label{sec:translation_e}

\subsection{Translating the Domain Language}

This translation is the same than in \ref{sec:trans_dom_ltl}.

\begin{itemize}
  %\item $\Pi$ and $\preceq$ can be represented by a trace $\pi$ of the same size. (with a $\next^{t_f}Tail$ or something like that)
  \item For all $a$ in $\Delta$ there is a propositional variable $a$.
  \item For all $f$ in $\Phi$ there are the propositional variables $f$, $initiation(f)$ and $termination(f)$.
\end{itemize}

\subsection{Axioms}

The advantage of equilibrium logic is a built-in completion of every atoms.
But this completion is defined by the presence of atoms in the consequences of implication.
The formula $A.1$ of \ref{sec:trans_ax_ltl} can't be used here, for the presence of $termination(f)$ in the head make his presence possible in unwarranted states.
So this axiomatisation is based on \label{sec:sec}.

\begin{description}
  \item[Inertia]
    For every $f$ in $\Phi$ we add in $F$:
    \begin{equation}\tag{$A.1$}\label{sec:sec}
      \alwaysF((f \trigger \neg termination(f)) \rightarrow f)
    \end{equation}
  \item[Initiation]
    For every $f$ in $\Phi$ we add in $F$:
    \begin{equation}\tag{$A.2$}
      \alwaysF(initiation(f)\rightarrow\wnext f)
    \end{equation}
%  \item Initiation Inertia $$\alwaysF(initiation(f_i)\rightarrow\next (termination(f_i) \mathbin{\bm{\mathsf{R}}} f_i))$$
%  \item Termination Inertia $$\alwaysF(termination(f_i)\rightarrow\next (initiation(f_i) \mathbin{\bm{\mathsf{R}}} \neg f_i))$$
\end{description}

On top of them, we also need an axiom to break the non-completion of the fluents.

\begin{description}
  \item[Fluents Initial State]
  For every $f$ in $\Phi$ we add in $F$:
  \begin{equation}\tag{$A.3$}
    f \lor \neg f
  \end{equation}
\end{description}

\subsection{Translating the Domain Definition}

This translation is the same than in \ref{sec:trans_def_ltl}.

\begin{description}
  \item[c-proposition]\footnote{c-proposition's translation also place initiation and termination point for the temporal logic}
  \begin{itemize}
    \item For every “$a$ initiates $f$ when $c$” in $\gamma$ we add in $F$:
    \begin{equation}\tag{$P_c.1$}
      \alwaysF((a\land c)\rightarrow initiation(f))
    \end{equation}
%    \begin{equation}\tag{$P_c.2$}
%      \alwaysF(initiation(f)\rightarrow (a\land c))
%    \end{equation}
    \item For every “$a$ terminates $f$ when $c$” in $\gamma$ we add in $F$:
    \begin{equation}\tag{$P_c.3$}
      \alwaysF((a\land c)\rightarrow termitranslation_e0nation(f))
    \end{equation}
%    \begin{equation}\tag{$P_c.4$}
%      \alwaysF(termination(f)\rightarrow (a\land c))
%    \end{equation}
  \end{itemize}
  \item[h-proposition] $ $
  \begin{itemize}
    \item For every “$f$ holds-at $t_i$” in $\eta$ we add in $F$:
    \begin{equation}\tag{$P_h.1$}
      \next^{i}f
    \end{equation}
    \item For every “$\neg f$ holds-at $t_i$” in $\eta$ we add in $F$:
    \begin{equation}\tag{$P_h.2$}
      \next^{i} \neg f
    \end{equation}
  \end{itemize}
  \item[t-proposition] $ $
  \begin{itemize}
    \item For every “$a$ happens-at $t_i$” in $\tau$ we add in $F$:
    \begin{equation}\tag{$P_t.1$}
      \next^{i}a
    \end{equation}
%    \item For all $a$ in $\Delta$ and all $t_i$ in $\Pi$, if “$a$ happens-at $t_i$” is not in $\tau$ we add in $F$:
%    \begin{equation}\tag{$P_t.2$}
%      \next^{i}\neg a
%    \end{equation}
%      This is needed to ensure completion of the actions.
  \end{itemize}
\end{description}

Since we are in Equilibrium Logic, we do not need to reinforce completion of the actions and their effects.

\subsection{Alternative Axiomatizations}

In this section, I wanted to explore alternative ways to express the axioms of $E$.

\subsubsection{Iteratifs Axioms}\label{sec:iter}

In the the Discrete Event Calculus, we only describe a time-point depending of the previous one.

\begin{itemize}
  \item
    For every $f$ in $\Phi$ we add in $F$:
    \begin{equation}\tag{$IA.1$}
      \alwaysF((f \land \neg termination(f)) \rightarrow f)
    \end{equation}
  \item
    For every $f$ in $\Phi$ we add in $F$:
    \begin{equation}\tag{$IA.2$}
      \alwaysF((\neg f \land \neg initiation(f)) \rightarrow \neg f)
    \end{equation}
\end{itemize}

% “…”
