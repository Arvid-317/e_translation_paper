\section{Action Langage $\mathcal{E}$}\label{sec:e}

\theoremstyle{definition}
\newtheorem{definition}{Definition}[section]

\subsection{Domain langage and domain description}

\begin{definition}[domain language]
  A \emph{domain language} is a tuple $\langle T,A,F\rangle$. $T$ is a non-empty set of natural numbers called timepoints, such as $\forall t \in [0..|T|), t \in T$.
  $A$ is a non-empty set of action constants and $F$ is a non-empty set of fluent constants.
\end{definition}

\begin{definition}[fluent literal]
  A \emph{fluent literal} of $\mathcal{E}$ is an expression either of the form $f$ or of the form $\neg f$, where $f \in F$.
\end{definition}

\begin{definition}[condition]
  A \emph{condition} of $E$ is a set of fluent literal.
\end{definition}

\begin{definition}[c-proposition]
  A \emph{c-proposition} in $\mathcal{E}$ is an expression either of the form
  $$a \text{ \textbf{initiates} } f \text{  \textbf{when} } c$$
  or of the form
  $$a \text{ \textbf{terminates} } f \text{  \textbf{when} } c$$
  where $f$ is a fluent literal, $a$ is an action constant and $c$ is a condition.
\end{definition}

\begin{definition}[h-proposition]
  A \emph{h-proposition} in $\mathcal{E}$ is an expression of the form
  $$a \text{  \textbf{happens-at} } t$$
  where $a$ is an action constant and $t$ is a timepoint.
\end{definition}

\begin{definition}[t-proposition]
  A \emph{t-proposition} in $\mathcal{E}$ is an expression of the form
  $$l \text{  \textbf{holds-at} } t$$
  where $l$ is an fluent literal and $t$ is a timepoint.
\end{definition}

\begin{definition}[domain description]
  A \emph{domain description} in $\mathcal{E}$ is a set of c-proposition, h-proposition and t-proposition.
\end{definition}

\subsection{Interpretation and model}

\begin{definition}[interpretation]
  An \emph{interpretation} in $\mathcal{E}$ is a mapping
  $$H:F,T \mapsto \{ \text{true;false} \}$$
\end{definition}

\begin{definition}[point satisfaction]
  Given a condition $c$ and a timepoint $t$, an interpretation $H$ \emph{satisfies $c$ at $t$} if
  \begin{itemize}
    \item for each fluent constant $f\in c$, $H(f,t)=\text{true}$
    \item for each fluent literal of the form $\neg f\in c$, $H(f,t)=\text{false}$
  \end{itemize}
\end{definition}

\begin{definition}[initiation/termination point]
  Let $H$ be an interpretation of $\mathcal{E}$, let $D$ be a domain description, let $f$ be a fluent and let $t$ be a timepoint. $t$ is an \emph{initiation-point} (respectively \emph{termination-point}) for $f$ in $H$ relative to $D$ iff there is an action $a$ such that
  \begin{enumerate}
    \item there is in $D$ both an h-proposition of the form $a \text{  \textbf{happens-at} } t$ and a c-proposition of the form $a \text{ \textbf{initiates} } f \text{  \textbf{when} } c$ (respectively $a \text{ \textbf{terminates} } f \text{  \textbf{when} } c$)
    \item $H$ satisfies $c$ at $t$.
  \end{enumerate}
\end{definition}

\begin{definition}[model]
  Given a domain description $D$ in $\mathcal{E}$, an interpretation $H$ of $\mathcal{E}$ is a \emph{model} of $D$ iff, for every fluent constants $f$ and every timepoints $t$, $t_1$ and $t_3$ such as $t_1 < t_3$, the following properties hold:
  \begin{enumerate}
    \item If there is no initiation-point or termination-point $t_2$ for $f$ in $H$ relative to $D$, with $t_1 \leq t_2 < t_3$, then $H(f,t_1)=H(f,t_3)$.
    \item If $t_1$ is an initiation-point for $f$ in $H$ relative to $D$, and there is no termination-point $t_2$ for $f$ in $H$ relative to $D$, with $t_1 < t_2 < t_3$, then $H(f,t_3)=\text{true}$.
    \item If $t_1$ is an termination-point for $f$ in $H$ relative to $D$, and there is no initiation-point $t_2$ for $f$ in $H$ relative to $D$, with $t_1 < t_2 < t_3$, then $H(f,t_3)=\text{false}$.
    \item For all t-proposition in $D$ of the form $f \text{  \textbf{holds-at} } t$, $H(f,t)=\text{true}$.
    \item For all t-proposition in $D$ of the form $\neg f \text{  \textbf{holds-at} } t$, $H(f,t)=\text{false}$.
  \end{enumerate}
\end{definition}

\subsection{Example: The gun}

\begin{itemize}
  \item $\mathcal{E}=\langle T,A,F\rangle$
  \item $T=\{0..1\}$
  \item $A=\{shoot\}$
  \item $F=\{loaded;dead\}$.
  \item $D$ is composed of all the following propositions:
  \begin{itemize}
    \item $shoot \text{ \textbf{initiates} } dead \text{  \textbf{when} } \{loaded\}$
    \item $shoot \text{ \textbf{terminates} } loaded \text{  \textbf{when} } \{\}$
    \item $shoot \text{  \textbf{happens-at} } 0$
    \item $dead \text{  \textbf{holds-at} } 1$
  \end{itemize}
\end{itemize}

There is three models of $D$ :
\begin{enumerate}
  \item \begin{itemize}
    \item $H(loaded,0)=H(loaded,1)=\text{false}$
    \item $H(dead,0)=H(dead,1)=\text{true}$
  \end{itemize}
  \item \begin{itemize}
    \item $H(loaded,0)=\text{true}$, $H(loaded,1)=\text{false}$
    \item $H(dead,0)=H(dead,1)=\text{true}$
  \end{itemize}
  \item \begin{itemize}
    \item $H(loaded,0)=\text{true}$, $H(loaded,1)=\text{false}$
    \item $H(dead,0)=\text{false}$, $H(dead,1)=\text{true}$.
  \end{itemize}
\end{enumerate}
