\section{Action Langage $\mathcal{E}$}\label{sec:e}

\theoremstyle{definition}
\newtheorem{definition}{Definition}[section]

\subsection{Domain langage and domain description}

\begin{itemize}
  \item A \emph{domain language} consist of a non-empty set of action constants and a non-empty set of fluent constants.
  \item A \emph{fluent literal} is either a fluent constant or his negation.
  \item A \emph{condition} is a conjunction of fluent literal.
  \item A \emph{c-proposition} is an expression of the form
    \begin{equation}\label{c1}\tag{$p_{c1}$}
    	a \text{ \textbf{initiates} } f \text{ \textbf{when} } c
    \end{equation}
    or of the form
    \begin{equation}\label{c2}\tag{$p_{c2}$}
    	a \text{ \textbf{terminates} } f \text{ \textbf{when} } c
    \end{equation}
    where $f$ is a fluent constant, $a$ is an action constant and $c$ is a condition.
  \item A \emph{h-proposition} is an expression of the form
    \begin{equation}\label{h}\tag{$p_h$}
      a \text{ \textbf{happens-at} } t
    \end{equation}
    where $a$ is an action constant and $t \in \mathbb{N} \cup \{0\}$.
  \item A \emph{t-proposition} is an expression of the form
    \begin{equation}\label{t}\tag{$p_t$}
      l \text{ \textbf{holds-at} } t
    \end{equation}
    where $l$ is an fluent literal and $t \in \mathbb{N} \cup \{0\}$.
  \item A \emph{domain description} is a set of c, h and t-propositions.
\end{itemize}

\subsection{Interpretation and model}

In this section, satisfaction $\vDash$ is the common $TEL$ satisfaction, as described in \cite{cakascsc18a}.

\begin{itemize}
%  \item Given a conjunction of fluent literal $c$ and $t \in \mathbb{N}$, an interpretation $N,t \vDash c$ if
%    \begin{itemize}
%      \item for each fluent constant $f\in c$, $N,t \vDash f$
%      \item for each fluent literal $\neg f\in c$, $N,t \nvDash f$
%    \end{itemize}
  \item Given a trace $N$, a c-proposition \ref{c1} and a h-proposition \ref{h}, $t$ is an \emph{initiation point of $f$} if $N,t \vDash c$ (with $t$ as in \ref{h} and $c$ as in \ref{c1}).
  \item Given a trace $N$, a c-proposition \ref{c2} and a h-proposition \ref{h}, $t$ is an \emph{termination point of $f$} if $N,t \vDash c$ (with $t$ as in \ref{h} and $c$ as in \ref{c2}).
  \item Given a domain description $D$, a trace $N$ is a \emph{model} of $D$ iff, for every fluent constants $f$ the following properties hold:
  \begin{enumerate}
    \item For all $t,t'\in [0..\lambda)$ with $t<t'$, if there is no initiation point $t_i$ of $f$ such as $t \leq t_i < t'$, and there is no termination point $t_t$ of $f$ such as $t \leq t_t < t'$, then $N,t \vDash f \leftrightarrow N,t' \vDash f$.
    \item For all $t,t'\in [0..\lambda)$ with $t<t'$, if $t$ is an initiation point of $f$ and there is no termination point of $f$ called $t_t$ such as $t < t_t < t'$, then $N,t' \vDash f$.
    \item For all $t,t'\in [0..\lambda)$ with $t<t'$, if $t$ is an termination point of $f$ and there is no initiation point of $f$ called $t_i$ such as $t < t_i < t'$, then $N,t' \nvDash f$.
    \item For all t-proposition \ref{t} in $D$, $N,t \vDash l$ (with $t$ and $l$ as in \ref{t}).
  \end{enumerate}
\end{itemize}

\subsection{Example: The gun}

\begin{itemize}
  \item $\mathcal{E}=\langle \{shoot\},\{loaded;dead\}\rangle$
  \item $D$ is composed of all the following propositions:
  \begin{itemize}
    \item $shoot \text{ \textbf{initiates} } dead \text{ \textbf{when} } \{loaded\}$
    \item $shoot \text{ \textbf{terminates} } loaded \text{ \textbf{when} } \{\}$
    \item $shoot \text{ \textbf{happens-at} } 0$
    \item $dead \text{ \textbf{holds-at} } 1$
  \end{itemize}
\end{itemize}
