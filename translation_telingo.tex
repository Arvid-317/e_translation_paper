\section{Translation into $telingo$}\label{sec:translation_telingo}

This section will describe a translation of the Action Language E into telingo.

The goal is to start from a domain language $E=<\Pi,\preceq,\Delta,\Phi>$ and a domain description $D=<\gamma,\eta,\tau>$, and to get a logic program $F$ such that all assignment of $F$ is a stable model of it if and only if the corresponding interpretation $H$ of $E$ is a model of $D$.

This translation is an adaptation of the iterative axiomatisation proposed in \ref{sec:iter}.

\subsection{Axioms}

\begin{description}
  \item[Common Inertia]
    For every $Fl$ in $\Phi$ we add in the $dynamic$ part of $F$:
    \begin{lstlisting}
      Fl :- 'Fl, not 'term(Fl).
    \end{lstlisting}
  \item[Initiation]
    For every $Fl$ in $\Phi$ we add in the $dynamic$ part of $F$:
    \begin{lstlisting}
      Fl :- 'init(Fl).
    \end{lstlisting}
  \item[Fluents Initiation]
    For every $Fl$ in $\Phi$ we add in the $initial$ part of $F$:
    \begin{lstlisting}
      {Fl}.
    \end{lstlisting}
\end{description}

\subsection{Translating the Domain Definition}

\begin{description}
  \item[c-proposition] $ $
    \begin{itemize}
    \item For every “$Ac$ initiates $Fl$ when $C$” in $\gamma$ we add in the $always$ part of $F$:
    \begin{lstlisting}
      init(Fl) :- Ac, C.
    \end{lstlisting}
    \item For every “$Ac$ terminates $Fl$ when $C$” in $\gamma$ we add in the $always$ part of $F$:
    \begin{lstlisting}
      term(Fl) :- Ac, C.
    \end{lstlisting}
  \end{itemize}
  \item[h-proposition] $ $
  \begin{itemize}
    \item For every “$Fl$ holds-at $t_i$” in $\eta$ we add in the $initial$ part of $F$:
    \begin{lstlisting}
      Fl'.
      \end{lstlisting}
    with the number of $'$ defined by $t_i$.
    \item For every “$\neg Fl$ holds-at $t_i$” in $\eta$ we add in the $initial$ part of $F$:
    \begin{lstlisting}
      not Fl'.
      \end{lstlisting}
    with the number of $'$ defined by $t_i$.
  \end{itemize}
  \item[t-proposition] $ $
  \begin{itemize}
    \item For every “$Ac$ happens-at $t_i$” in $\tau$ we add in the $initial$ part of $F$:
    \begin{lstlisting}
      Ac'.
      \end{lstlisting}
    with the number of $'$ defined by $t_i$.
  \end{itemize}
\end{description}

\section{Example of a telingo encoding}

We take the same example than before.
This is what the translation give us.

\begin{lstlisting}
#program initial.

switch'.
switch'''.

{light}.
not light''''.

#program always.

init(light) :- switch, not light.
term(light) :- switch, light.

#program dynamic.

light :- 'init(light).
light :- 'light, not 'term(light).
\end{lstlisting}
