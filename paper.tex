\documentclass[a4paper]{article}

\usepackage[utf8]{inputenc}
\usepackage{amsmath}
\usepackage{amssymb}
\usepackage{microtype}
\usepackage{biblatex}
\usepackage{url}\urlstyle{tt}
\usepackage{comments}
\usepackage{listings}

\RequirePackage{bm}
\RequirePackage{textcomp}
\RequirePackage{upgreek}

\IfFileExists{outline.tex}{\input{outline}}{}

\newcommand{\next}{\text{\rm \raisebox{-.5pt}{\Large\textopenbullet}}}  % {{\ensuremath{\circ}}}
\newcommand{\previous}{\text{\rm \raisebox{-.5pt}{\Large\textbullet}}}  % {{\ensuremath{\bullet}}}
\newcommand{\wnext}{\ensuremath{\widehat{\next}}}
\newcommand{\wprevious}{\ensuremath{\widehat{\previous}}}
\newcommand{\alwaysF}{\ensuremath{\square}}
\newcommand{\alwaysP}{\ensuremath{\blacksquare}}
\newcommand{\eventuallyF}{\ensuremath{\Diamond}}
\newcommand{\eventuallyP}{\ensuremath{\blacklozenge}}
\IfFileExists{outline.tex}
        {\newcommand{\until}{\ensuremath{\mathbin{\mbox{\outline{$\bm{\mathsf{U}}$}}}}}}
        {\newcommand{\until}{\ensuremath{\mathbin{\bm{\mathsf{U}}}}}}
\IfFileExists{outline.tex}
        {\newcommand{\release}{\ensuremath{\mathbin{\mbox{\outline{$\bm{\mathsf{R}}$}}}}}}
        {\newcommand{\release}{\ensuremath{\mathbin{\bm{\mathsf{R}}}}}}
\IfFileExists{outline.tex}
        {\newcommand{\while}{\ensuremath{\mathbin{\mbox{\outline{$\bm{\mathsf{W}}$}}}}}}
        {\newcommand{\while}{\ensuremath{\mathbin{\bm{\mathsf{W}}}}}}
\newcommand{\since}{\ensuremath{\mathbin{\bm{\mathsf{S}}}}}
\newcommand{\trigger}{\ensuremath{\mathbin{\bm{\mathsf{T}}}}}
\IfFileExists{outline.tex}
        {\newcommand{\finally}{\ensuremath{\mbox{\outline{$\bm{\mathsf{F}}$}}}}}
        {\newcommand{\finally}{\ensuremath{\bm{\mathsf{F}}}}}
\newcommand{\initially}{\ensuremath{\bm{\mathsf{I}}}}

\newcommand{\stp}{\ensuremath{\uptau}}

\newcommand{\dalways}[1]{\ensuremath{[#1]\,}}                        % \DBox
\newcommand{\deventually}[1]{\ensuremath{\langle#1\rangle\,}}        % \DDia
\newcommand{\drel}[2]{\ensuremath{{\parallel}{#1}{\parallel}^{#2}}}  % \Rel

\newcommand{\intervc}[2]{[#1..#2]}
\newcommand{\intervo}[2]{[#1..#2)}
\newcommand{\ointerv}[2]{(#1..#2]}
\newcommand{\rangec}[3]{#1 \in \intervc{#2}{#3}}
\newcommand{\rangeo}[3]{#1 \in \intervo{#2}{#3}}
\newcommand{\orange}[3]{#1 \in \ointerv{#2}{#3}}


\bibliography{krr,procs}

\begin{document}

\title{Translating Action Langage $E$ into $LTL_f$ and $TEL_f$}

\author{%
  Etienne Tignon\\
  University of Potsdam, Germany
}

\maketitle

\section{Action Langage E}
This paper describe a translation of the Action Language E (which closely resemble Event Calculus in multiples ways) into $LTL_f$ and $TEL_f$.

The goal is to start from a domain language $E=\langle T,A,F\rangle$ containing :
\begin{itemize}
  \item $T$ a set of $n$ natural numbers from $0$ to $n-1$, representing timepoints.
  \item $A$ a set of actions.
  \item $F$ a set of fluents.
\end{itemize}
and a domain description $D$ containing the propositions:
\begin{itemize}
  \item $a$ initiates $f$ when $c$.\footnote{$c$ is a set of fluent and negated fluent}
  \item $a$ terminates $f$ when $c$.\footnotemark[1]
  \item $a$ happens-at $t$.
  \item $f$ holds-at $t$.
  \item $\neg f$ holds-at $t$.
\end{itemize}

and to get a set of temporal propositions $S_a$ and formulas $S_f$, such that all formulas of $S_f$ are true for a finite traces if and only if the corresponding interpretation $H$ of $E$ is a model of $D$.

\section{Translation}\label{sec:translation}

This section will describe a translation of the Action Langage E (which closely ressemble Event Calculus in multiples ways) into Temporal Logic.
To be more precise, into $LTL_f$.

The goal is to start from a domaine langage $E=<\Pi,\preceq,\Delta,\Phi>$ and a domain description $D=<\gamma,\eta,\tau>$, and to get a set of temporal propositions $A$ and formulas $F$, such that all formulas of $F$ are true for a finite traces $\pi$ if and only if the corresponding interpretation $H$ of $E$ is a model of $D$.

\subsection{Naive approach}

\subsubsection{Translating the Domain Language}

To make the translation, we need to start from a domain langage for wich $\Pi$ is a discrete set of timepoint, and $\preceq$ is a total ordering of $\Pi$.

The set $A$ of propositional atoms is composed of the atoms $a_i$, $f_i$, $initiation(f_i)$ and $termination(f_i)$.

\begin{itemize}
  \item $\Pi$ and $\preceq$ can be represented by a trace $\pi$ of the same size. (with a $\circ^{t_f}Tail$ or something like that)
  \item For all $A_i$ in $\Delta$ there is a propositional variable $a_i$.
  \item For all $F_i$ in $\Phi$ there is a propositional variable $f_i$ and two propositional variables $initiation(f_i)$ and $termination(f_i)$.
\end{itemize}

\subsubsection{Axiomes}

\begin{itemize}
  \item Common Inertia
  $$\Box(f_i\rightarrow\circ (termination(f_i) \mathbin{\bm{\mathsf{R}}} f_i))$$
  $$\Box(\neg f_i\rightarrow\circ (initiation(f_i) \mathbin{\bm{\mathsf{R}}} \neg f_i))$$
  \item Initiation Inertia $$\Box(initiation(f_i)\rightarrow\circ f_i)$$
  \item Termination Inertia $$\Box(termination(f_i)\rightarrow\circ \neg f_i)$$
%  \item Initiation Inertia $$\Box(initiation(f_i)\rightarrow\circ (termination(f_i) \mathbin{\bm{\mathsf{R}}} f_i))$$
%  \item Termination Inertia $$\Box(termination(f_i)\rightarrow\circ (initiation(f_i) \mathbin{\bm{\mathsf{R}}} \neg f_i))$$
\end{itemize}

\subsubsection{Translationg the Domain Definition}

\begin{itemize}
  \item c-proposition\footnote{c-proposition's translation also contains the definition of initiation and termination point for the temporal logic}

  $$\Box(a_i\bigwedge c_i)\rightarrow initiation(f_i)$$ or $$\Box(a_i\bigwedge c_i)\rightarrow termination(f_i)$$
  \item h-proposition

  $$\circ^{t_i}f_i$$ or $$\circ^{t_i} \neg f_i$$
  \item t-proposition

  $$\circ^{t_i}a_i$$
\end{itemize}

\subsubsection{Example}

Let's take a domaine langage $E=<\Pi,\preceq,\Delta,\Phi>$ with :
\begin{itemize}
  \item $\Pi$ composed of 5 timepoints $t_0$ to $t_4$
  \item $\Delta$ only composed of the action $switch$
  \item $\Phi$ only composed of the action $light$
\end{itemize}

Let's take a domain description $D=<\gamma,\eta,\tau>$ with :
\begin{itemize}
  \item $\gamma$ composed of two c-proposition :
  \begin{itemize}
    \item $switch$ initiates $light$ when $\neg light$.
    \item $switch$ initiates $\neg light$ when $light$.
  \end{itemize}
  \item $\eta$ composed of two h-proposition :
  \begin{itemize}
    \item $switch$ happens-at $t_1$.
    \item $switch$ happens-at $t_3$.
  \end{itemize}
  \item $\tau$ composed of no t-proposition.
\end{itemize}

Once translated we get :
\begin{itemize}
  \item The lengh of the trace $lengh(\pi)=5$.
  \item The set of propositions $A$ composed of $switch$, $light$, $initiation(light)$ and $termination(light)$.
  \item The set of formulas $F$ composed of :
  \begin{itemize}
    \item The formulas taken from the propositions :
    \item $\Box(switch\bigwedge \neg light)\rightarrow initiation(light)$
    \item $\Box(switch\bigwedge light)\rightarrow termination(light)$
    \item $\circ^{1}switch$
    \item $\circ^{3}switch$
    \item The formulas taken from the inertia axiomes :
    \item $\Box(light\rightarrow\circ (termination(light) \mathbin{\bm{\mathsf{R}}} light))$
    \item $\Box(\neg light\rightarrow\circ (initiation(light) \mathbin{\bm{\mathsf{R}}} \neg light))$
    \item $\Box(initiation(light)\rightarrow\circ light)$
    \item $\Box(termination(light)\rightarrow\circ \neg light)$
  \end{itemize}
\end{itemize}

We can see that the original domain description had two models:
\begin{itemize}
  \item $H(light,t_0)=H(light,t_1)=H(light,t_4)=true$ and $H(light,t_2)=H(light,t_3)=false$
  \item $H(light,t_0)=H(light,t_1)=H(light,t_4)=false$ and $H(light,t_2)=H(light,t_3)=true$
\end{itemize}

Likewise, the resulting set of formulas only has two traces that respect them:
\begin{itemize}
  \item $\{light\}\{light,switch,termination(light)\}\{\}\{switch,initiation(light)\}\{light\}$
  \item $\{\}\{switch,initiation(light)\}\{light\}\{light,switch,termination(light)\}\{\}$
\end{itemize}

\section{Example}\label{sec:example}

Let's take a domain language $E=<\Pi,\preceq,\Delta,\Phi>$ with :
\begin{itemize}
  \item $\Pi$ composed of 5 timepoints $t_0$ to $t_4$
  \item $\Delta$ only composed of the action $switch$
  \item $\Phi$ only composed of the action $light$
\end{itemize}

Let's take a domain description $D=<\gamma,\eta,\tau>$ with :
\begin{itemize}
  \item $\gamma$ composed of two c-proposition :
  \begin{itemize}
    \item $switch$ initiates $light$ when $\neg light$.
    \item $switch$ terminates $light$ when $light$.
  \end{itemize}
  \item $\eta$ composed of two h-proposition :
  \begin{itemize}
    \item $switch$ happens-at $t_1$.
    \item $switch$ happens-at $t_3$.
  \end{itemize}
  \item $\tau$ composed of no t-proposition.
\end{itemize}

Once translated we get :
\begin{itemize}
  %\item The length of the trace $length(\pi)=5$.
  \item The set of propositions $A$ composed of $switch$, $light$, $initiation(light)$ and $termination(light)$.
  \item The set of formulas $F$ composed of :
  \begin{itemize}
    \item The formulas taken from the propositions :
    \item $\alwaysF((switch\bigwedge \neg light)\rightarrow initiation(light))$
    \item $\alwaysF(initiation(light)\rightarrow (switch\bigwedge \neg light))$
    \item $\alwaysF((switch\bigwedge light)\rightarrow termination(light))$
    \item $\alwaysF(termination(light)\rightarrow (switch\bigwedge light))$
    \item $\next^{1}switch$
    \item $\next^{3}switch$
    \item The formulas taken from the inertia axioms :
    \item $\alwaysF(light\rightarrow (termination(light) \release light))$
    \item $\alwaysF(\neg light\rightarrow (initiation(light) \release \neg light))$
    \item $\alwaysF(initiation(light)\rightarrow\wnext light)$
    \item $\alwaysF(termination(light)\rightarrow\wnext \neg light)$
  \end{itemize}
\end{itemize}

We can see that the original domain description had two models:
\begin{itemize}
  \item $H(light,t_0)=H(light,t_1)=H(light,t_4)=true$ and $H(light,t_2)=H(light,t_3)=false$
  \item $H(light,t_0)=H(light,t_1)=H(light,t_4)=false$ and $H(light,t_2)=H(light,t_3)=true$
\end{itemize}

Likewise, the resulting set of formulas only has two traces that respect them:
\begin{itemize}
  \item $\{light\}\{light,switch,termination(light)\}\{\}\{switch,initiation(light)\}\{light\}$
  \item $\{\}\{switch,initiation(light)\}\{light\}\{light,switch,termination(light)\}\{\}$
\end{itemize}

\section{Translation into $TEL$}\label{sec:translation_e}

\subsection{Translating the Domain Language}

This translation is the same than in \ref{sec:trans_dom_ltl}.

\begin{itemize}
  %\item $\Pi$ and $\preceq$ can be represented by a trace $\pi$ of the same size. (with a $\next^{t_f}Tail$ or something like that)
  \item For all $a$ in $\Delta$ there is a propositional variable $a$.
  \item For all $f$ in $\Phi$ there are the propositional variables $f$, $initiation(f)$ and $termination(f)$.
\end{itemize}

\subsection{Axioms}

The advantage of equilibrium logic is a built-in completion of every atoms.
But this completion is defined by the presence of atoms in the consequences of implication.
The formula $A.1$ of \ref{sec:trans_ax_ltl} can't be used here, for the presence of $termination(f)$ in the head make his presence possible in unwarranted states.
So this axiomatisation is based on \label{sec:sec}.

\begin{description}
  \item[Inertia]
    For every $f$ in $\Phi$ we add in $F$:
    \begin{equation}\tag{$A.1$}\label{sec:sec}
      \alwaysF(((\previous \neg termination(f)) \since f) \rightarrow f)
    \end{equation}
  \item[Initiation]
    For every $f$ in $\Phi$ we add in $F$:
    \begin{equation}\tag{$A.2$}
      \alwaysF(initiation(f)\rightarrow\wnext f)
    \end{equation}
%  \item Initiation Inertia $$\alwaysF(initiation(f_i)\rightarrow\next (termination(f_i) \mathbin{\bm{\mathsf{R}}} f_i))$$
%  \item Termination Inertia $$\alwaysF(termination(f_i)\rightarrow\next (initiation(f_i) \mathbin{\bm{\mathsf{R}}} \neg f_i))$$
\end{description}

On top of them, we also need an axiom to break the non-completion of the fluents.

\begin{description}
  \item[Fluents Initial State]
  For every $f$ in $\Phi$ we add in $F$:
  \begin{equation}\tag{$A.3$}
    f \lor \neg f
  \end{equation}
\end{description}

\subsection{Translating the Domain Definition}

This translation is the same than in \ref{sec:trans_def_ltl}.

\begin{description}
  \item[c-proposition]\footnote{c-proposition's translation also place initiation and termination point for the temporal logic}
  \begin{itemize}
    \item For every “$a$ initiates $f$ when $c$” in $D$ we add in $F$:
    \begin{equation}\tag{$P_c.1$}
      \alwaysF((a\land c)\rightarrow initiation(f))
    \end{equation}
%    \begin{equation}\tag{$P_c.2$}
%      \alwaysF(initiation(f)\rightarrow (a\land c))
%    \end{equation}
    \item For every “$a$ terminates $f$ when $c$” in $D$ we add in $F$:
    \begin{equation}\tag{$P_c.3$}
      \alwaysF((a\land c)\rightarrow termination(f))
    \end{equation}
%    \begin{equation}\tag{$P_c.4$}
%      \alwaysF(termination(f)\rightarrow (a\land c))
%    \end{equation}
  \end{itemize}
  \item[h-proposition] $ $
  \begin{itemize}
    \item For every “$f$ holds-at $t_i$” in $D$ we add in $F$:
    \begin{equation}\tag{$P_h.1$}
      \next^{i}(\neg f\rightarrow \bot)
    \end{equation}
    \item For every “$\neg f$ holds-at $t_i$” in $D$ we add in $F$:
    \begin{equation}\tag{$P_h.2$}
      \next^{i}(f\rightarrow \bot)
    \end{equation}
  \end{itemize}
  \item[t-proposition] $ $
  \begin{itemize}
    \item For every “$a$ happens-at $t_i$” in $D$ we add in $F$:
    \begin{equation}\tag{$P_t.1$}
      \next^{i}a
    \end{equation}
%    \item For all $a$ in $\Delta$ and all $t_i$ in $\Pi$, if “$a$ happens-at $t_i$” is not in $D$ we add in $F$:
%    \begin{equation}\tag{$P_t.2$}
%      \next^{i}\neg a
%    \end{equation}
%      This is needed to ensure completion of the actions.
  \end{itemize}
\end{description}

Since we are in Equilibrium Logic, we do not need to reinforce completion of the actions and their effects.

\subsection{Alternative Axiomatizations}

In this section, I wanted to explore alternative ways to express the axioms of $E$.

\subsubsection{Iteratifs Axioms}\label{sec:iter}

In the the Discrete Event Calculus, we only describe a time-point depending of the previous one.

\begin{itemize}
  \item
    For every $f$ in $\Phi$ we add in $F$:
    \begin{equation}\tag{$IA.1$}
      \alwaysF(\previous(f \land \neg termination(f)) \rightarrow f)
    \end{equation}
  \item
    For every $f$ in $\Phi$ we add in $F$:
    \begin{equation}\tag{$IA.2$}
      \alwaysF(\previous(\neg f \land \neg initiation(f)) \rightarrow \neg f)
    \end{equation}
\end{itemize}

% “…”

\section{Translation into $telingo$}\label{sec:translation_telingo}

This section will describe a translation of the Action Language E into telingo.

The goal is to start from a domain language $E=<\Pi,\preceq,\Delta,\Phi>$ and a domain description $D=<\gamma,\eta,\tau>$, and to get a logic program $F$ such that all assignment of $F$ is a stable model of it if and only if the corresponding interpretation $H$ of $E$ is a model of $D$.

This translation is an adaptation of the iterative axiomatisation proposed in \ref{sec:iter}.

\subsection{Axioms}

\begin{description}
  \item[Common Inertia]
    For every $Fl$ in $\Phi$ we add in the $dynamic$ part of $F$:
    \begin{lstlisting}
      Fl :- 'Fl, not 'term(Fl).
    \end{lstlisting}
  \item[Initiation]
    For every $Fl$ in $\Phi$ we add in the $dynamic$ part of $F$:
    \begin{lstlisting}
      Fl :- 'init(Fl).
    \end{lstlisting}
  \item[Fluents Initiation]
    For every $Fl$ in $\Phi$ we add in the $initial$ part of $F$:
    \begin{lstlisting}
      {Fl}.
    \end{lstlisting}
\end{description}

\subsection{Translating the Domain Definition}

\begin{description}
  \item[c-proposition] $ $
    \begin{itemize}
    \item For every “$Ac$ initiates $Fl$ when $C$” in $\gamma$ we add in the $always$ part of $F$:
    \begin{lstlisting}
      init(Fl) :- Ac, C.
    \end{lstlisting}
    \item For every “$Ac$ terminates $Fl$ when $C$” in $\gamma$ we add in the $always$ part of $F$:
    \begin{lstlisting}
      term(Fl) :- Ac, C.
    \end{lstlisting}
  \end{itemize}
  \item[h-proposition] $ $
  \begin{itemize}
    \item For every “$Fl$ holds-at $t_i$” in $\eta$ we add in the $initial$ part of $F$:
    \begin{lstlisting}
      Fl'.
      \end{lstlisting}
    with the number of $'$ defined by $t_i$.
    \item For every “$\neg Fl$ holds-at $t_i$” in $\eta$ we add in the $initial$ part of $F$:
    \begin{lstlisting}
      not Fl'.
      \end{lstlisting}
    with the number of $'$ defined by $t_i$.
  \end{itemize}
  \item[t-proposition] $ $
  \begin{itemize}
    \item For every “$Ac$ happens-at $t_i$” in $\tau$ we add in the $initial$ part of $F$:
    \begin{lstlisting}
      Ac'.
      \end{lstlisting}
    with the number of $'$ defined by $t_i$.
  \end{itemize}
\end{description}

\section{Example of a telingo encoding}

We take the same example than before.
This is what the translation give us.

\begin{lstlisting}
#program initial.

switch'.
switch'''.

{light}.
not light''''.

#program always.

init(light) :- switch, not light.
term(light) :- switch, light.

#program dynamic.

light :- 'init(light).
light :- 'light, not 'term(light).
\end{lstlisting}

%\begin{abstract}
TODO
\end{abstract}

%\section{Introduction}\label{sec:introduction}

%\section{Background}\label{sec:background}


%\section{Approach}\label{sec:approach}


%\section{Discussion}\label{sec:discussion}

\url{potassco.org}


\section{Todo}

\begin{itemize}
  \item Reduce the trace size? Do we need all timepoints, or only the ones with actions?
  \item $LTL_f$ into $TEL$ ... into $telingo$
  \item actions with duration (which probably will need metric temporal logic)
\end{itemize}

\printbibliography{}

\end{document}
